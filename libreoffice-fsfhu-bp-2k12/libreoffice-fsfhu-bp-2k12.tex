\documentclass[a4paper]{report}

\usepackage{alltt}
\usepackage{graphicx}
\usepackage{url}
\usepackage[utf8]{inputenc}
\usepackage[magyar]{babel}

\title{Hogyan készítsünk új LibreOffice Writer funkciókat?}
\author{Vajna Miklós <vmiklos@suse.cz>}

\begin{document}

\tableofcontents

\chapter{Vajna Miklós: Hogyan készítsünk Writer funkciókat?}

\section{Bevezetés}

Nem csak a funkciók számítanak. Mielőtt belevetnénk magunkat a funkciók
készítésének rejtelmeibe, ne felejtsük el, hogy nem ez az egyetlen módja a
LibreOffice projekt segítésének. Segíthetjük a felhasználókat (levelező lista,
fórum, IRC), tevékenykedhetünk a minőség javításán (hibajelentések készítése,
nem megerősített hibák reprodukálása), útmutatókat írhatunk, a design csoport
az inkább csak a fejlesztéshez értő fejlesztőknek ad tanácsot jó felhasználói
felületek kialakításához. Az infrastruktúrát üzemeltető önkéntesek nélkül a
többiek nem tudnának tevékenykedni.

De még egy fejlesztőre váró kihívások közül is csak egy az új funkciók
készítése: sok időt visz el a hibák javítása, tesztelés, a létező kód
átrendezése, új önkéntesek segítése, dokumentálás. Ezek mind legalább annyira
fontosak, mint az új funkciók fejlesztése. Ennek ellenére, ez a cikk a
továbbiakban kizárólag ezzel a témával foglalkozik.

Mielőtt továbbhaladnánk, nézzünk két példát új funkciókra, amik a LibreOffice
4.0-ban fognak bemutatkozni:

\begin{itemize}
\item Korábban csak egy adott ponthoz fűzhettünk megjegyzéseket. Az új
\emph{hozzászólás szövegtartományhoz} funkció lehetővé teszi, hogy -- egy
kijelölés után -- olyan megjegyzést készítsünk, mely eltérő kezdő
és végpozícióval rendelkezik.
\item A \emph{más fejléc/lábléc az első oldalon} funkció lehetővé teszi, hogy
egyetlen oldalstílust használva ne csak a jobb/bal oldalon lehessen külön
fejléc/lábléc, hanem az első oldalon is.
\item
\end{itemize}

Mindkét funkciót régóta kérik a felhasználók -- ezt tekinthetjük általános
szabálynak is: egy, a 80-as évek óta fejlesztett irodai szoftvercsomagba
érdemes meggondolni, milyen új funkcióra van szükség, csak akkor érdemes
nekikezdeni valami újnak ha arra valós igény van.

Mindezek után tekintsük át azt a nyolc lépést, amit minden új funkció
elkészítése során végig érdemes követnünk:

\begin{enumerate}
\item dokumentum modell
\item UNO API
\item megjelenítés (layout)
\item szűrők
\item felhasználói felület
\item tesztek
\item dokumentáció
\item specifikáció
\end{enumerate}

A cikk hátralevő részében ezeket a lépéseket részletezzük.

\section{Dokumentum modell}

Azt lehetne gondolni, hogy egy funkció fejlesztése során csak ezzel kell
foglalkozni. A Writer klasszikus Modell-View-Controller paradigmában
gondolkozik. Egy dokumentum egy \texttt{SwDoc} osztálynak feleltethető meg. Ezen
belül az építőkockánk a paragrafusok (\texttt{SwNode} osztály). A forráskódban
ezek az \texttt{sw/source/core/} könyvtár alatt találhatóak.

Érdekesség, hogy nincs külön szöveges csomópont a
szövegtartományoknak\footnote{Olyan szakasz egy paragrafuson belül, ahol a
karakterek tulajdonságai is -- pl. betűtípus -- megegyeznek.}, helyette egy
\texttt{SwpHints} osztályba gyűjtve minden paragrafuson belül változó
tulajdonsághoz egy \texttt{SwTxtAttr} tartozik, kezdeti pozíció, végpozíció és
érték információkkal.

Ezen kívül még sok a kivétel, külön kollekcióban (nem paragrafusokhoz csatolva)
találhatók meg például az oldalstílusok (\texttt{SwPageDesc} osztály). Az „első
oldali fejléc” funkcióhoz például ehhez az osztályhoz kellett az
\texttt{aMaster} és \texttt{aLeft} tagok mellé egy \texttt{aFirst} tagot is
felvenni.

Aki szeret példákon keresztül tanulni, annak érdemes az

\begin{verbatim}
SW_DEBUG=1 ./soffice.bin
\end{verbatim}

opcióval indítania a Writert, ilyenkor a Shift-F12 a \texttt{nodes.xml} file-ba
kiírja a jelenleg aktuális dokumentummodellt.

Másik lehetőség, ha UNO-n keresztül, makróból szeretnénk végigiterálni a paragrafusokon:

\begin{verbatim}
enum = ThisComponent.Text.createEnumeration
para = enum.NextElement
para = enum.NextElement
xray para
\end{verbatim}

Így egy tetszőleges paragrafus tulajdonságait tekinthetjük meg.

\section{UNO API}

Makrókból nem férhetünk hozzá közvetlenül a Writer belső dokumentum
modelljéhez, ehhez annak UNO API-ját kell használnunk. Ezért fontos, hogy az új
funkciónkat elérhetővé tegyük ezen a publikus API-n keresztül is.

Ez egyébként is hasznos, mivel ha makróból már tudjuk írni/olvasni az új
funkciót, akkor a felhasználói felület elkészítése előtt is már tudjuk azt
tesztelni. Ezen kívül az UNO alapú (ld. később) szűrők csak az UNO API-n
elérhető funkciókat tudják menteni, valamint a tesztesetek is az UNO API-t
használják legtöbbször. Az Writer UNO API kódját az
\texttt{sw/source/core/unocore/} alatt találjuk.

Nézzünk példákat! Ha a környezetérzékeny térközöket szeretnénk bekapcsolni egy paragrafusra:

\begin{verbatim}
enum = ThisComponent.Text.createEnumeration
para = enum.NextElement
xray para.ParaContextMargin
para.ParaContextMargin = True
\end{verbatim}

Ha szeretnénk, hogy az első oldalon más fejléc/lábléc legyen:

\begin{verbatim}
oDefault = ThisComponent.StyleFamilies.PageStyles.Default
oDefault.HeaderIsOn = True
oDefault.HeaderIsShared = False
oDefault.FirstIsShared = False
\end{verbatim}

Ha megjegyzést szeretnénk csatolni egy szövegtartományhoz:

\begin{verbatim}
oTextField =
oDoc.createInstance("com.sun.star.text.TextField.Annotation")
oTextField.TextRange.String = "Tartalom"
oDoc.Text.insertTextContent(oCurs, oTextField, True)
\end{verbatim}

\section{Layout}

A layout célja megjeleníteni azt, ami eddig csak a dokumentum modelljében volt
elrejtve, azaz a „View” az MVC-ből. A Writer esetében ennek is saját dokumentum
modellje van, aminek az építőkövei a keretek. Egy megnyitott dokumentum egy
\texttt{SwRootFrm}-nek felel meg, egy oldal egy \texttt{SwPageFrm}-nek, majd
ezen belül egy paragrafus egy \texttt{SwTxtFrm}-nek. Egy modell elem több
layout elemnek is megfelelhet, gondoljunk csak arra az esetre ha egy fejléc
több oldalon is megjelenik, vagy egy paragrafus átfolyik a következő oldalra.
Ehhez az \texttt{1:N} megfeleltetéshez a Writer a az \texttt{SwClient} /
\texttt{SwModify} mechanizmust használja: egy kliens csak egy kiszolgálóhoz
lehet regisztrálva, viszont a kiszolgálóknak lehet több kliense, melyeken az
\texttt{SwClientIter} osztállyal lehet iterálni.

Ha példát szeretnénk látni a layout felépítésére, a dokumentum modelljéhez
hasonló módon megtehetjük, csak F12-re lesz szükségünk a Shift-F12 helyett.

\section{Szűrők}

Próbáljuk meg túlélni az újraindítást! Ehhez a szűrők úgy segítenek bennünket,
hogy az exporterek egy \texttt{SwDoc} példányt mentenek egy streamre, az
importerek pedig egy stream alapján felépítenek egy (eredetileg üres)
\texttt{SwDoc}-ot.

Alapból minden szűrő „alien”, ami azt jelenti, hogy ismert, hogy információt
vesztenek a konvertálás során. Egyetlen kivétel az ODF szűrő, amire emiatt
„own”-ként hivatkoznak. Ennél a szűrőnél garantált a veszteségmentesség, ezért
is kell kiterjeszteni \emph{minden} új funkció implementálásakor.

A szűrők lehetnek UNO-alapúak (pl. DOCX import, RTF import), lehetnek
beépítettek (belső sw API-t használják, pl. DOC import, DOC/DOCX/RTF
export), végül lehetnek kevertek, mint amilyen az ODF szűrő (főleg UNO, de kis
részben belső). Ezek forráskódjai az \texttt{sw/source/filter/},
\texttt{writerfilter/}, ill. az \texttt{xmloff/} alatt találhatóak.

\section{Felhasználói felület}

Eljutottunk oda, hogy a felhasználók számára is élvezhetővé tudjuk tenni a
munkánkat. A felhasználói felület a színfalak mögött lehet régi vagy új.

A jelenlegi ablakok nagy része még a régi formátumot használja, tervezői
vélhetőleg „ne használjunk egeret az UI elkészítéséhez!” felkiáltással
tervezhették. Tipikusan egy \texttt{.src}, \texttt{.hrc} és egy \texttt{.cxx}
file készült minden ablakhoz. Minden elem rögzített pozícióval / mérettel
rendelkezett. Például:

\begin{verbatim}
CheckBox CB_SHARED_FIRST
{
HelpID ="svx:CheckBox:RID_SVXPAGE_HEADER:CB_SHARED_FIRST";
Pos = MAP_APPFONT ( 12 , 46 ) ;
Size = MAP_APPFONT ( 152 , 10 ) ;
Text [ en-US ] = "Same content on first page" ;
};
\end{verbatim}

Számos probléma van ezzel a megoldással, például ha új elemet adunk az ablak
közepéhez, az összes többi vezérlőt le kell mozgatni kézzel, vagy a gomb
szélessége a leghosszabb fordítást kell figyelembe vegye, stb.

Ezzel szemben sokkal előremutatóbb Caolán új Glade-alapú
megoldása\footnote{\url{http://conference.libreoffice.org/talks/}}, bár ha csak
egy-egy új vezérlőt adunk létező ablakokhoz, egyelőre ritkán találkozunk velük.

Az UI gyakran közös, nem lehet alkalmazás-specifikus, így az UI és a dokumentum modell közötti interakcióra a feltalált megoldás az \texttt{SfxItemSet} lett. Ha egy jelölőnégyzetet szeretnénk \texttt{SfxItemSet}-be tenni:

\begin{verbatim}
aSet.Put(SfxBoolItem(nWSharedFirst,
    aCntSharedFirstBox.IsChecked()));
\end{verbatim}

Ez után kell az \texttt{SfxItemSet}-ből a dokumentum modellbe (jelen esetben \texttt{SwPageDesc}) rakni:

\begin{verbatim}
rPageDesc.ChgFirstShare(((const SfxBoolItem&)
    rHeaderSet.Get(SID_ATTR_PAGE_SHARED_FIRST)).GetValue());
\end{verbatim}

Ellenkező irányban is hasonló megoldásra lesz szükségünk. Erre számos létező
példa található: a közös kód az \texttt{svx/source/dialog/} és a \texttt{cui/}
alatt, a Writer-specifikus kód az \texttt{sw/source/ui/} alatt.

\section{Tesztesetek}

Miért ismételnénk meg régi hibákat, ha választhatunk újak közül is? :-) A tesztesetek ebben segítenek minket.

A LibreOffice-ban unitcheckeket (\emph{minden} részleges modul fordítás végén
lefutnak), slowcheckeket (minden teljes fordítás végén futnak), valamint
subsequentcheckeket (külön \texttt{make subsequentcheck}-re futnak csak)
használunk. Új funkciók esetén legtöbbször egy új slowcheck-et készítünk: ez
hozzáférést ad az UNO API-hoz, viszont minden fejlesztő gépén lefut, így széles
tesztelést biztosít.

Tesztelés során legtöbbször a dokumentum modellt (UNO API) vagy a layoutot (XML
kiírás + XPath kifejezés kiértékelése) vizsgáljuk. Ez utóbbira egy példa:

\begin{verbatim}
CPPUNIT_ASSERT_EQUAL(OUString("Első fejléc"),
    parseDump("/root/page[1]/header/txt/text()"));
\end{verbatim}

Import / export tesztek esetén először minimális reprodukáló dokumentumot
készítünk. A tesztelés során vagy importálunk, vagy egy import-export-import
szekvenciát hajtunk végre, attól függően, hogy mit akarunk tesztelni. Ez azért
jó, mert így a tesztelendő dokumentumot mindig file-ban tárolhatjuk (nem kódból
kell felépíteni), valamint a tesztelést is egy létező \texttt{SwDoc} példányon
tudjuk elvégezni, nem pedig egy valamilyen formátumban elmentett file-t kell
vizsgáljunk.

Az elkészült (importált) dokumentum modelljét az UNO API-val könnyen
vizsgálhatjuk. Pl. ha valamilyen nem-ASCII karakter probléma volt, sokszor elég
csak a dokumentum hosszát (karaktereinek) számát vizsgálni:

\begin{verbatim}
CPPUNIT_ASSERT_EQUAL(6, getLength());
\end{verbatim}

A teszt megírása után a teszt futtatása következik. Ha sikeres, érdemes
visszavonni a javítást vagy funkciót, és megbizonyosodni arról, hogy a teszt
hibák jelez, vagy is a teszt jó. Ha ez stimmel, eldobhatjuk a visszavonást és
összevonhatjuk a két commitot, így később nem kell keresni, hogy melyik
módosítást tesztelte a teszteset.

A Writer tesztjeit az \texttt{sw/qa/} alatt keressük.

\section{Dokumentáció}

A felhasználói dokumentáció a súgó, ami F1-re jelenik meg. Új tartalmat ehhez
úgy érdemes hozzáadni, hogy a felhasználói felületen kikeressük a jövendőbeli
szakasz szomszédos részeit, majd \texttt{git grep}-pel rákeresünk a létező
tartalmakra a \texttt{helpcontent2/} alatt.

Fontos, hogy míg normál kód módosításkor elég a fordítás a változtatásunk
teszteléséhez (a \texttt{linkoo}-nak köszönhetően), így fordítás után egy
telepítés is kell (\texttt{make dev-install}) mielőtt kipróbáljuk annak
eredményét. További trükk, hogy minden paragrafusnak kell egy egyedi azonosító
a fordításokat segítendő, de ez csak egy file-on belül kell egyedi legyen,
tehát egyszerűen másoljunk le egy létezőt, majd tetszőleges algoritmust
választva tegyük egyedivé\footnote{Egyszerű inkrementálás, ínyenceknek
kvadratikus próba tökéletesen megfelel. :-)}.

A fejlesztői dokumentációra mindössze annyi a követelmény, hogy minden új osztálynak
legyen legalább egy egysoros \texttt{doxygen} leírása, hogy mit csinál --
preferáltan minden publikus metódusra is, de ez már nem követelmény.

\section{Specifikáció}

Azért nem ezzel kezdünk egy új funkciót, mert mindent ODF-be mentünk, ami egy nyílt szabvány,
és csak az kerül bele az ODF-be, amit már 3 független implementáció
életképesnek mutat. Ennek megfelelően alapból a LibreOffice egy \emph{ODF 1.2
Extended} formátumba ment, ami tartalmazza a saját kiegészítéseinket, majd
ezeket javaslatként eljuttatjuk az OASIS-hoz.

Egy ilyen módosítási kérelemnek tartalmaznia kell, hogy mi a motiváció a
változtatásra, a szabvány szövegéhez valamint a RELAX NG sémához milyen
változtatásokat szeretnénk eszközölni, végül részleteznie kell a változtatás
várt hatását (visszafele kompatibilitás kezelése, stb).

\section{Elérhetőségek}

A szerző ezúton is elnézést kér, hogy sok -- a cikkben szereplő -- témát
csak érintőlegesen említett, csak a Writer belső működéséről vastag könyvet
lehetne írni, a cél leginkább a figyelemfelkeltés volt.  Az alábbi linkek
további kérdések esetén remélhetőleg segítséget nyújtanak.

\begin{itemize}
\item LibreOffice honlap: \url{http://libreoffice.org/}
\item A diák és ezen cikk elérhetősége: \url{http://vmiklos.hu/odp/}
\end{itemize}

\end{document}
